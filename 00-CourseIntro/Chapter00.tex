\documentclass[t]{beamer}

% Load general definitions
\input{../defs/preamble.tex}

% Specific definitions
\title[]{Experiment Design for Computer Sciences}
\subtitle[]{00 - Course Intro}
\author[]{Lecturer: Claus Aranha\\{\footnotesize Slides by Felipe Campelo, with modifications}}
\institute{Computer Science Department}
\date{\scriptsize Tsukuba, 2016}

\begin{document}

% cover page
\setbeamertemplate{footline}{}
\begin{frame}
\begin{flushright}
%\includegraphics[width=.25\textwidth]{../figs/principal_completa3_ufmg}
\end{flushright}
  \titlepage
  \begin{tikzpicture}[remember picture,overlay]
  \node[anchor=south east,xshift=-5pt,yshift=122pt] at (current page.south east) {\tiny Version 2.11.T};
  \node[anchor=south west,yshift=0pt] at (current page.south west) {\includegraphics[width=.15\textwidth]{../figs/by-nc-sa.png}};
  \end{tikzpicture}  
\end{frame}

% quotation page
  \begin{frame}[b]
		\frametitle{}
\begin{columns}[T]
\column{0.8\textwidth}
\flushright{\small ``\textit{The combination of some data and an aching desire\\for an answer does not ensure that a reasonable answer\\can be extracted from a given body of data.}''\\\ \\John W. Tukey (1915 -- 2000)\\American mathematician\\}
\column{0.2\textwidth}
\begin{tikzpicture}[remember picture,overlay]
\node[anchor=south east,yshift=15pt,xshift=0pt] at (current page.south east) {\includegraphics[width=\textwidth]{../figs/JohnTukey.jpg}};
\end{tikzpicture}
\end{columns}
\vskip .5em
\lfr{Image source: \url{http://www-history.mcs.st-andrews.ac.uk/Mathematicians/Tukey.html}}
\end{frame}

% Main slides
\begin{ftst}
{Course Overview}
{Objectives}
\bitems To develop advanced skills in designing experiments, defining and testing hypotheses, and performing statistical data analyses within one's field of interest;
\spitem By the end of this course, the student should be able to:
\bitems Plan experiments related to his/her work;
\item Perform appropriate statistical analyses of the data obtained from the experiment;
\item Develop sound conclusions based on the available data;
\item Identify the problems and limitations of his/her own experiments, and suggest improvements;
\item Perform critical interpretations of other experimental methodologies and results reported in the literature.
\eitem\eitem
\end{ftst}

%=====

\begin{ftst}
{Course Overview}
{Course Structure}

\bitems Lectures (10 weeks): discussions about several aspects and
techniques for design and analyses of experiments. Theory, application
examples and \textit{computational case studies};
\spitem One Team Project/Presentation;
\spitem One Individual Report;
\spitem One Case Study;
\eitem
\end{ftst}

%=====

\begin{ftst}
{Course Overview}
{Course Structure}
\begin{block}{Evaluation criteria}
	\begin{center}
		\small
		\begin{tabular}{cccc} \hline
			\textbf{Item}	& \textbf{Type}	&Value\\
			\hline
			Case studies		& Individual Report &40\\
			Research Report		& Individual Report &30\\
			Research Project	& Report and presentation &30\\
			\hline
		\end{tabular}
	\end{center}
\end{block}

\begin{block}{Other relevant Information}
	\bitems Lectures slides, example R files, data, etc. available at \\
	{\small \url{https://git.io/vVpvC/}}
	\spitem MANABA site for submitting reports and questions.\\ 
        Code: 5718071
	\spitem Software/services used: R ({\scriptsize\url{http://cran.r-project.org/}}),\\GitHub ({\scriptsize\url{http://github.com/}}).
	\eitem
\end{block}
\end{ftst}

%=====

\begin{ftst}
{Course Overview}
{Course Bibliography}
\textbf{Main}:\\
{\footnotesize
\bitems \textit{Lecture Notes on Design and Analysis of Experiments}. Online: \url{http://git.io/vVpvC} \structure{Version 2.11.t}; Creative Commons BY-NC-SA 4.0.
\item D.C. Montgomery, G.C. Runger (2010), \textit{Applied Statistics and Probability for Engineers}, John Wiley \& Sons.
\item Michael J. Crawley (2007), \textit{The R Book}, Wiley.
\eitem}
\vhalf
\textbf{Additional}:
{\footnotesize
\bitems D.C. Montgomery (2012), \textit{Design and Analysis of Experiments}, John Wiley \& Sons.
\item B. Caffo (2015), \textit{Statistical inference for data science}, LeanPub - {\small\url{https://leanpub.com/LittleInferenceBook/}}
\item J.J. Faraway (2002), \textit{Practical Regression and Anova using R} - {\small\url{http://goo.gl/ewMWL}}
\item D. Wiens (2005), \textit{Introduction to Design and Analysis of Experiments} - {\small\url{http://goo.gl/hZXg1}}
\eitem}
\end{ftst}

%=====

\begin{ftst}
{Course Overview}
{Required / Desired background}
This is a course on \textit{applied} experimental design and
analysis. As such, a large portion of the course is dedicated to case
studies in which the student will design experiments, collect
(simulated) data, perform inference and report his or her analysis.
\vone
It is \textbf{strongly reccomended} that the student should complete
the free online course \textit{R
  Programming}\footnote{{\scriptsize\url{https://www.coursera.org/course/rprog}}}
\textbf{before the end of the second week} of the semester (except if
the student is already fluent with R).
\vone
It is also \textbf{strongly reccomended} that the student should
complete the free online course \textit{Reproducible
  Research}\footnote{{\scriptsize\url{https://www.coursera.org/course/repdata}}}
\textbf{before the end of the first month} of the semester (except if
the student is already fluent with writing reports using R Markdown).
\end{ftst}

%=====

\begin{ftstf}{About this material}{Conditions of use and referencing}

\centering\footnotesize This work is a derivative work of \\
\centering\footnotesize Felipe Campelo (2015), \textit{Lecture Notes on Design and Analysis of Experiments}.\\
Online: {\scriptsize\url{https://github.com/fcampelo/Design-and-Analysis-of-Experiments}}\\
Version 2.11;
\vone
\centering\footnotesize It is licensed under the Creative Commons CC BY-NC-SA 4.0 license\\(Attribution Non-Commercial Share Alike International License version 4.0).\\
\vhalf
\url{http://creativecommons.org/licenses/by-nc-sa/4.0/}\\
\vone
\footnotesize I'd prefer if you referenced the original work above, but this work can be referenced as:\\
\footnotesize \flushleft Claus Aranha (2016), \textit{Lecture Notes on Experiment Design for Computer Sciences}.\\
Online: {\scriptsize\url{https://github.com/caranha/Design-and-Analysis-of-Experiments}}\\
Version 2.11.t; Creative Commons BY-NC-SA 4.0.\\

\begin{tikzpicture} [remember picture,overlay]
\node[anchor=south,yshift=0pt] at (current page.south){ \includegraphics[width=.2\textwidth]{../figs/CCsomerights.png}};
\end{tikzpicture}
\end{ftstf}

%=====

\begin{ftst}{About this material}{Acknowledgments}
\footnotesize Special thanks to:\\
\vhalf
\bitems Prof. Felipe Campelo, PhD. from UFMG, for organizing the
original material, and making it available under a Creative Commons
License
\eitem
\end{ftst}

\end{document}
