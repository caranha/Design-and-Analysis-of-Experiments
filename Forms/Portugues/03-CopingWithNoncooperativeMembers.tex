\documentclass{letter}
\usepackage{color}
\usepackage{xcolor}
\usepackage{url}
\usepackage{graphicx}
\usepackage{fancyhdr}
\usepackage[utf8]{inputenc}
\usepackage[T1]{fontenc}

\signature{\ }
\address{\ \\
EEE933 - Planejamento e Análise de Experimentos\\
Prof. Felipe Campelo}

\fancypagestyle{firstpage}{\fancyhf{}\fancyhead[R]{\includegraphics[height=.8in, keepaspectratio=true]{../../figs/ENG_ELETRICA_ufmg.png}}}
\fancypagestyle{empty}{\fancyhf{}\fancyhead[R]{\includegraphics[height=.8in, keepaspectratio=true]{../../figs/ENG_ELETRICA_ufmg.png}}}

\begin{document}
\pagestyle{empty}
\begin{letter}{}
\date{}
\opening{\vskip -5em}

\begin{center}
{\sc Como lidar com caroneiros e preguiçosos\footnote{Adaptado de B. Oakley \textit{et al.}, ``Turning Student Groups into Effective Teams'', 2004.}}
\end{center}

Na maioria dos casos, seus colegas estão tão interessados quanto você em aprender e se dedicar às atividades do curso. Contudo, você ocasionalmente encontrará alguém que cause problemas, o que pode ser bastante frustrante. O objetivo deste documento é fornecer alguns conselhos práticos sobre como lidar com estas situações.

Para ilustrar este tipo de situação, vamos imaginar uma situação hipotética: suponha que você tenha sido alocado para um grupo com três outros colegas: vamos chamá-los de Clint, Lee, e Eli. Clint é tranquilo - não é particularmente habilidoso na resolução dos problemas, mas é aplicado e proativo, buscando encontrar as respostas e ativamente procurando auxílio externo (por exemplo, do professor ou dos monitores) para questões nas quais o grupo tenha dificuldade. Lee é irritante -- não é uma má pessoa, mas simplesmente não se esforça para fazer um bom trabalho. Ele usualmente entrega sua parte do trabalho apenas parcialmente realizada, e confessa que não conseguiu fazer porque ``viajou'' ou ``teve um churrasco no final de semana''.

O último membro -- Eli -- é um problemão. Eis algumas das coisas que ele já aprontou:

\begin{itemize}
\item Quando vocês tentaram agendar as reuniões do grupo no início do semestre, Eli simplesmente vetou todas as alternativas, afirmando que estava ocupado demais;
\item Eli quase nunca entrega sua parte das tarefas, e quando o faz o trabalho está quase sempre errado. É absolutamente claro que ele não dedicou o tempo necessário à tarefa, e simplesmente rabiscou algo que tivesse a aparência de um trabalho.
\item Eli nunca responde às mensagens do grupo. Quando você pergunta, ele nega que tenha recebido a mensagem. De forma similar, ele está sempre ``ocupado demais para responder aos e-mails''.
\item Eli nunca comparece às reuniões do grupo -- ele sempre promete que vai aparecer, mas nunca cumpre.
\item Embora Eli escreva relativamente bem, ele jamais consegue apresentar um bom relatório -- está sempre perdendo os rascunhos, não relê seu trabalho, esquece de inserir tabelas ou figuras, ou completa os relatórios de forma mal-acabada (por exemplo, escrevendo as equações à mão). Você parou de atribuir trabalho a ele, pois fica com medo de perder o \textit{deadline} dos relatórios.
\item Eli está frequentemente reclamando de seu excesso de trabalho, alta carga de aulas, livros ruins, e do quão picaretas são os professores. No início você até sentiu uma certa pena, mas depois de um tempo começou a pensar que ele talvez esteja te enrolando.
\item Eli sempre fala de forma agressiva e autoconfiante quando vocês tentam conversar com ele sobre estes problemas. Ele está firmemente convencido de que os problemas são sempre culpa dos outros, e não dele. Ele fala de forma tão assertiva que às vezes você até pensa que ele talvez tenha razão.
\end{itemize}

Seu grupo finalmente decide que não suporta mais esta situação, e marca uma conversa com o professor. Este, por sua vez, chama Eli para uma conversa junto com o restante do grupo, mas Eli declara de forma aparentemente sincera e convincente que ele realmente não tinha entendido o que o grupo esperava dele. O professor declara que a razão deste problema deve ser uma falha de comunicação do grupo. Ele percebe que você e Clint estão claramente irritados, enquanto que Eli aparenta estar um pouco confuso e triste, mas absolutamente inocente. Neste cenário, seria fácil para o professor concluir (de forma incorreta) que o grupo como um todo é disfuncional e que todos os membros são culpados -- e que provavelmente Eli é o menos culpado de todos.

Resumindo: \textit{você e seus colegas de equipe acabam tendo que carregar o grupo nas costas. Eli vai receber as mesmas notas que todos, mas sem precisar dividir o trabalho. No processo, ele ainda conseguiu ``queimar o seu filme'' com o professor da disciplina.}

\vskip 1em
\noindent\textbf{\sc Onde o grupo errou: Absorção}

Pode-se dizer que este grupo foi um \textit{absorvedor}. Desde o início o grupo absorveu o problema toda vez que Eli fez algo errado, e talvez tenha se orgulhado de concluir o trabalho a qualquer custo. \textbf{Caroneiros se aproveitam deste tipo de atitude}. Quanto mais ``legal'' você é, mais o caroneiro conseguirá se aproveitar da situação para te forçar a fazer o trabalho dele -- tanto na universidade quanto na vida profissional. Ao absorver os problemas do membro caroneiro, você está de forma indireta contribuindo para que esta pessoa ache cada vez mais que é aceitável receber crédito pelo trabalho alheio.

\vskip 1em
\noindent\textbf{\sc Como o grupo deveria ter agido: Reflexão}

É importante que o comportamento disfuncional do caroneiro seja refletido de volta para ele, de forma que o próprio caroneiro pague o preço por suas atitudes, ao invés do resto do grupo. Nunca aceite acusações, culpa, ou críticas de um caroneiro. Mantenha seu próprio senso da realidade, independentemente do que o caroneiro te acusar (algo que é mais fácil de falar do que de fazer). \textbf{Deixe claro que você tem um limite para o tipo de comportamento que está disposto a aceitar}. Mostre estes limites de forma clara, e haja de forma consistente. Por exemplo, eis algumas das coisas que o grupo poderia ter feito:

\begin{itemize}
\item Quando Eli não foi capaz de concordar com um horário para as reuniões, mesmo depois que alternativas foram sugeridas, você poderia ter avaliado se havia indícios de comportamento caroneiro. A atitude de Eli foi grosseira, egoísta, ou ele aparentou simplesmente não ter paciência para discutir ``estas bobagens''? Esse seria um bom momento para deixar claro ao mesmo que ele deve encontrar um horário compatível ou ir conversar com o professor da disciplina.
\item Se Eli não entrega sua parte do trabalho, seu nome não deve ser incluído no relatório (Exceção: se o colega de equipe já demonstrou anteriormente que é trabalhador e dedicado, é completamente apropriado ajudá-lo em caso de algum imprevisto). Em cursos onde os grupos possuem mecanismos para ``demitir'' um membro não-cooperativo, esta pode ser uma opção a ser discutida com o professor.
\item Se Eli entregar sua parte do trabalho de forma mal-feita ou claramente incompleta, você deve informá-lo de que o mesmo não contribuiu de forma significativa para o trabalho, e que seu nome não será incluído no relatório do trabalho. Independente do que o caroneiro argumentar, mantenha sua atitude! Caso Eli se torne grosseiro ou abusivo, converse com o professor e mostre a ele o material entregue por Eli ao grupo. Faça isto na primeira vez em que Eli entregar um trabalho porco, antes que ele tenha tido a oportunidade de se aproveitar do grupo -- não depois de dois ou três trabalhos, quando você já está frustrado e nervoso com a situação.
\item Defina seus limites desde o início, e defina limites altos. Caroneiros possuem uma capacidade impressionante de detectar até onde podem ir sem se dar mal.
\item Se Eli não responde às mensagens ou e-mails do grupo e não comparece às reuniões, não perca mais tempo tentando contactá-lo. Quando se lida com caroneiros, é particularmente importante utilizar e-mails, uma vez que os mesmos constituem um registro por escrito das tentativas de contato. Enviar os e-mails com cópia para o professor (ou, em ambientes profissionais, para o supervisor ou gerente) frequentemente gera resultados surpreendentemente rápidos.
\item Lembre-se que a única pessoa capaz de resolver os problemas de Eli é o próprio Eli. Você não conseguirá mudar a atitude do caroneiro, mas você sempre pode mudar a sua própria atitude, de forma que ele não se aproveite do seu esforço. Se você faz todo o trabalho por ele, não haverá qualquer incentivo para que o mesmo mude suas atitudes.
\end{itemize}

Pessoas com o perfil de Eli muitas vezes são ótimos manipuladores. Até que você perceba que está enrolando o grupo e agindo como um caroneiro, o semestre já terá terminado e ele estará livre para continuar parasitando outros grupos. É importante interromper este tipo de comportamento disfuncional já no início dos semestre, antes que o caroneiro se aproveite de você e sua equipe.

\vskip 1em
\noindent\textbf{\sc Lee, o preguiçoso}

Já falamos bastante de Eli, mas ainda não abordamos o problema de Lee. Apesar deste último ter se juntado ao grupo na batalha contra o comportamento irracional de Eli, ele também não tem contribuído de forma satisfatória para o bom desempenho do grupo.

Em geral, a melhor forma de lidar com um membro preguiçoso é similar àquela descrita para o caroneiro: estabeleça expectativas firmes e bem definidas para cada um, e mantenha sua atitude em relação a estas expectativas. Embora os preguiçosos não sejam tão manipuladores quanto os caroneiros, é usual que ele também testem o grupo para ver até que ponto podem enrolar os colegas. Se os membros do grupo não estabelecem limites aos comportamentos considerados aceitáveis, então parte da culpa pelo problema é dos próprios membros.

\vskip 1em
\noindent\textbf{\sc Mas eu não gosto de conflitos}

Se você é uma pessoa gentil que sempre evita confrontações, ter que lidar com um caroneiro ou um preguiçoso pode ser uma oportunidade de aprender a importante habilidade da \textit{firmeza}. Seja paciente consigo mesmo enquanto aprende -- nas primeiras tentativas você provavelmente se pegará pensando coisas do tipo ``mas se eu fizer isso ele/ela não vai mais me achar uma pessoa legal'', ``isso não vale o esforço''. Esta é uma reação normal, mas é importante continuar tentando manter a firmeza e desenvolver a capacidade de não permitir que outros se aproveitem do seu trabalho. No processo, você perceberá que terá mais tempo para sua família, amigos, ou para outras tarefas acadêmicas, uma vez que não precisará mais fazer o trabalho dos outros junto com o seu próprio.

\vskip 1em
\noindent\textbf{\sc Atitudes que facilitam a vida do caroneiro (e dificultam a sua)}

\begin{itemize}
\item Relutar em deixar que o caroneiro seja reprovado e eventualmente aprenda com seus erros;
\item Devoção exagerada à ideia do ``bem coletivo'', sem a compreensão de que outros podem simplesmente estar tirando vantagem dos demais;
\item Sempre querer fazer com que todos se sintam bem, mesmo que a um alto custo pessoal;
\item Achar que sempre precisa fazer mais que os outros;
\item Estar disposto a interpretar qualquer contribuição do caroneiro ou preguiçoso como um ``progresso'', por menor que seja esta contribuição;
\item Síndrome do mártir: a ideia de que somente você conseguiria se sacrificar tanto pelo caroneiro, que é alguém que ``precisa ser ajudado'';
\item Ser capaz de cooperar, mas não de delegar;
\item Ter a tendência de se sentir responsável pelos outros ao invés de por si mesmo.
\end{itemize}

\vskip 1em
\noindent\textbf{\sc Quando o trabalho todo sobra para você}

Assim que você percebe que os demais membros estão deixando o trabalho todo nas suas mãos -- ou fazendo um trabalho tão ruim que você acaba tendo que refazer tudo -- é importante tomar uma atitude. É comum que professores permitam solicitações de mudança de grupo. Se for este o caso, procure o professor e converse com ele.

\end{letter}
\end{document}
