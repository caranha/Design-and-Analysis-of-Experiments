\documentclass[t]{beamer}

\input{../../../defs/preamble.tex}
\usecolortheme{crane}

\title[]{Experiment Design for Computer Sciences}
\subtitle[]{Week 4 -- Pre-class Notes}
\author[]{Claus Aranha\\{\footnotesize \url{http://conclave.cs.tsukuba.ac.jp/}}}
\institute{Computer Science Department}
\date{\scriptsize Tsukuba, 2017}

\begin{document}

\section{Exercise 1 Review}
\subsection{Exercise 1 Review}
% cover page
\setbeamertemplate{footline}{}
\begin{frame}
  \titlepage
\end{frame}

\begin{ftst}
  {Exercise 1: Mini-experiment}{Thanks for submitting the first exercise}

  Many of the submissions were extremely interesting!

  \bigskip
  
  \begin{block}{Popular Themes}
    \begin{itemize}
    \item Physical exercise: 3
    \item Computer exercises: 2
    \item Interviews: 4
    \item Games: 2
    \item Cooking: 4
    \item Sleep: 2
    \item Online Data Collection: 1
    \item Biology: 1
    \end{itemize}
  \end{block}
\end{ftst}

\begin{ftst}
  {Exercise 1: Mini-experiment}{Warnings}
  Some common mistakes:
  \medskip
  
  \begin{block}{Forgetting Name, ID and Title}
    A very large number of students did not include their Names, ID or
    a title for the report... Why?
  \end{block}

  \begin{block}{Not performing an experiment}
    Quite a few students suggested an experiment, but did not perform
    it. The goal of the exercise was to {\bf execute} an experiment.
  \end{block}

  \begin{block}{Citing your sources}
    A few people used pictures or tables copied from other places
    without attribution. \alert{Don't do that!}
  \end{block}

\end{ftst}

\begin{ftst}
  {Exercise 1: Mini-experiment}{Grades}
  \begin{center}
    \includegraphics<1>[width=0.8\textwidth]{gradeshist}
    \includegraphics<2>[width=0.8\textwidth]{gradebreakdown}
  \end{center}
\end{ftst}

\begin{ftst}
  {Exercise 1: Mini-experiments}{Thoughts}
  \begin{itemize}
  \item Some very good reports;
  \item Almost all reports were active experiments (one observational);
  \item It is important to notice that experiments always bring
    questions that were not included in the original design;
  \item Hypothesis building in the ``question'' section (today's class!);
  \end{itemize}
\end{ftst}

\begin{ftst}
  {Chapter 5: Statistical Inference}{}
\end{ftst}

\section{Exercises 2 and 4}
\begin{ftst}
  {Report 2: Report on your own research}{Applying what you have learned}

  \begin{block}{Report Description}
    Write a description of {\bf your current/planned research work},
    focusing on the experiment design aspects learned in this course. Identify:
    \begin{itemize}
    \item Questions of interest and relevant hypothesis;
    \item Experiments and pre-experiments necessary;
    \item Factors and sources of uncertainty of the
      experiments, and write suggestions of how to deal with them;
    \item Assumptions of your experiments, and how to
      validate them;
    \end{itemize}
  \end{block}

  \vfill
  
  \begin{itemize}
  \item {\bf Executing an experiment} and {\bf data analysis} are not
    necessary.
  \item The report should be a PDF file with 2-4 pages maximum.
  \end{itemize}
\end{ftst}

\begin{ftst}
  {Report 4: Group Experiment}{Report 1's big sister}

  \begin{block}{Report Description}
    Choose an experiment idea, and perform all the steps in
    experimental research: Definition of the hypothesis, design of the
    experiment, data collection, analysis and description of the
    results. Prepare a report and presentation about your experiment.
  \end{block}

  \begin{itemize}
  \item Groups of \structure{up to 2 people}
  \item Using the same experiment idea as the first report is OK.
  \item Report will be graded on \alert{experiment
    quality}\footnote{and report/presentation quality too}
    \begin{itemize}
    \item Quality of the Hypothesis
    \item Quality of the Data collection
    \item Quality of the Numerical and Graphical Analysis
    \item Reproducibility of the results
    \end{itemize}
  \end{itemize}

  Later I will upload a report template to MANABA

  \vspace{.5cm}
  
\end{ftst}

\end{document}


