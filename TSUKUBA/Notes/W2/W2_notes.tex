\documentclass[t]{beamer}

\input{../../../defs/preamble.tex}
\usecolortheme{crane}

\title[]{Experiment Design for Computer Sciences}
\subtitle[]{Week 2 -- ``The Role of Experimentation'' Notes}
\author[]{Claus Aranha\\{\footnotesize \url{http://conclave.cs.tsukuba.ac.jp/}}}
\institute{Computer Science Department}
\date{\scriptsize Tsukuba, 2017}

\begin{document}

\section{Introduction}
\subsection{Lecturer}
% cover page
\setbeamertemplate{footline}{}
\begin{frame}
  \titlepage
\end{frame}

\section{Lecture Review}
\subsection{Overview}
% Main slides
\begin{ftst}
  {Lecture Overview}{What did we learn today?}

  Main ``take-aways'' from this lecture:

  \begin{itemize}
  \item What is a scientific experiment;
  \item Three types of experiment;
  \item Fundamental Principles and Concepts of an experiment;
  \end{itemize}
\end{ftst}

\subsection{My experiences}

\begin{ftst}
  {In real life...}{Let me share two experiences}
  
  These concepts of experimentalism will follow you through your life
  as a researcher.

  \vone

  Let us discuss a few recent experiences;
\end{ftst}

\begin{ftst}
  {In real life...}{Follow your curiosity}

  Student A was using a fine-tuning tool to select a value for a
  parameter in one algorithm, before beginning the ``true''
  experiment.

  \vone

  The tool suggested that the best value would be a very low
  value. {\bf However} the theory behind the algorithm suggested that
  high values would be better. This is strange!

  \vone

  First the student checked for bugs in his code first, and then the
  bibliography. He also learned that a few papers also suggested the low
  value, but without justification.

  \vone

  Based on these findings, we decided to start a new experiment to
  understand the influence of this parameter on the data we were
  studying, and compare that with the usual benchmarks.
\end{ftst}

\begin{ftst}
  {In real life...}{Kill your babies}

  Student B developed a new algorithm to solve an optimization
  problem. He work very hard to find the set of parameters that would
  result in the best performance for his algorithm.

  \vone

  He compared his result with two older algorithms. He had source code
  access to those other algorithms, so he ran the programs without
  changing any parameters.

  \vone

  Unfortunately, one of the parameters that was adjusted was the
  computational effort of the algorithm. So the proposed algorithm was
  allowed to spend 100 times more computational effort than the other
  two algorithms.

  \vone

  You must always make sure that all comparisons in your experiment
  are fair. This is specially important when setting parameters.
\end{ftst}

%\begin{ftst}
%  {In real life...}{Experiment EVERYTHING}
%  Minecraft Bug Hunting
%\end{ftst}


\section{Report}
\subsection{Report 1: Mini-Experiment}
\begin{ftst}
  {Report 1 -- Mini Experiment}{Outline}

  Based on the discussions of \emph{experimentation} in this lecture,
  I want you to think and execute a small experiment.

  \vone

  In this exercise:
  \begin{itemize}
    \item {\bf I do not require} mathematical/statistical
      rigour, but
    \item {\bf I do expect} some creativity and scientific
      honesty.
  \end{itemize}

  \vone

  \begin{block}{Example}
    Juliana and Claus disagree on whether bananas stay ``good'' longer
    inside or outside a fridge. They decide to settle the discussion
    with an experiment.
  \end{block}
\end{ftst}

\begin{ftst}
  {Report 1 -- Mini Experiment}
  {Example}

    \begin{block}{Example}
      Do bananas stay ``good'' longer inside or outside a fridge?
  \end{block}

    What do they want to know?
    
    \begin{itemize}
    \item Appearance of the bananas: Color and consistency.
    \item Quality of the bananas: Taste and texture.
    \end{itemize}

    How do they do the experiment?
    \begin{itemize}
    \item Buy 6 bananas.
    \item Two of them are put whole in the fridge;
    \item Two of them are put whole on top of the fridge (covered)
    \item Two of them are cut in half, and one is on the fridge, one
      is on top;
    \end{itemize}

    How is the result evaluated?
    \begin{itemize}
    \item Visually inspect the bananas every day for dark spots
    \item Open the bananas after 7 days.
    \item If they are not moldy, eat them.
    \end{itemize}
\end{ftst}

\begin{ftst}
  {Report 1 -- Mini Experiment}{Ideas}
  \begin{block}{}
    The bananas example is not very rigorous, and could be improved in
    many ways, but it illustrate one way how experimentation thought
    (some times called \emph{data-driven}) can be used in our lives)
  \end{block}

  \vone
  
  Other ideas:
  \begin{itemize}
  \item Is it faster to use \emph{memset} or \emph{assign} in C++?
  \item Do you get wetter if you walk or run in the rain?
  \item Does the ``Dry'' option in the bathroom really works?
  \item Do you get more ``likes'' on facebook with cat pictures or
    with memes?
  \item Do students pick up trash from the street more than
    non-students?
  \item etc...
  \end{itemize}
\end{ftst}

\begin{ftst}
  {Report 1 - Mini Experiment}{Report Outline}

  Think, execute and write a report on an experiment on any topic of
  your interest. The report structure must be (roughly) as follows:

  \vone

  \begin{itemize}
  \item {\bf The question}: What do you want to learn in your experiment? And why?
  \item {\bf The alternatives}: What are the possible answers for your
    question that you can think \emph{before you begin your
      experiment}?
  \item {\bf The experiment}: What experiment did you try to do to
    answer your question? What data would you collect from your experiment?
  \item {\bf The results}: What were the results of your experiment?
    What did you learn from it?
  \end{itemize}
\end{ftst}

\begin{ftst}
  {Report 1 - Mini Experiment}{Terms and Conditions}

  \begin{itemize}
  \item Deadline is May 8th (Monday after Golden Week).
  \item Submission is through Manaba (See assignments tab).
  \item Submission must be in PDF format, maximum 2 pages.
  \item This report is individual.
  \item This report will be graded on:
    \begin{itemize}
    \item How interesting is the question;
    \item How relevant is the experiment to the question;
    \item How insightful was the analysis of the result;
    \item How well written is the report
    \end{itemize}
  \item This report will {\bf not} be graded on:
    \begin{itemize}
    \item The actual result of the experiment;
    \item Statistical or mathematical rigour;
    \item Relevance of the question;
    \end{itemize}
  \end{itemize}
\end{ftst}

\end{document}


