\documentclass[t]{beamer}

% Load general definitions
\input{../../defs/notespreamble.tex}

% Specific definitions
\title[]{Experiment Design for Computer Sciences}
\subtitle[]{Week 04 -- Inferential Statistics -- Course Notes}
\author[]{Claus Aranha\\{\footnotesize \url{http://conclave.cs.tsukuba.ac.jp/}}}
\institute{Computer Science Department}
\date{2016-05-13 (updated \today)}

\begin{document}
\section{Outline}
\subsection{Outline}
\begin{frame}
  \maketitle
\end{frame}

\begin{frame}
  \frametitle{Course Notes/Warnings}

  \begin{alertblock}{Team Project -- Don't procastinate on your team project!}
    \begin{itemize}
      \item I have so far only received 1 team e-mail;
      \item We are almost at the halfway point of the course! 
    \end{itemize}
  \end{alertblock}

  \vfill

  Send me an e-mail (caranha@cs.tsukuba.ac.jp) with the names, student
  IDs, and mail addresses of your team members.

  \vfill

  If you send me the topic of your project, I would be very happy too.
\end{frame}

\section{Last Class}
\subsection{Last Class}

\begin{frame}
  \frametitle{Review of Last Class}

  \begin{itemize}
  \item \structure{Concepts of Statistics}
    \begin{itemize}
    \item Goats, and why the world sometimes does not follow our intuition
    \item Difference between \structure{sample} and \structure{population}
    \item The \structure{Central Limit Theorem}
    \end{itemize} 

    \vfill

  \item \structure{DAE Chapter 03 -- Point Estimators}
    \begin{itemize}
    \item Estimating values for the \structure{population parameters} from the \structure{sample parameters};
    \item \structure{Bias} and \structure{variance} of estimators;
    \end{itemize}
    
    \vfill

  \item \structure{DAE Chapter 04 -- Interval Estimators}
    \begin{itemize}
    \item \structure{Confidence Interval} for a parameter
    \item How to interpret the Confidence Interval;
    \end{itemize}
  \end{itemize}
\end{frame}

\begin{frame}
  \frametitle{Review of Last Class -- Examples} 

  In Manaba, there was a ``Chapter04.R'' file with example code to
  calculate Confidence Intervals in R (and plot them).

  \bigskip

  Please take a look at the code and apply the calculations to your
  own data. (If we have time, we will review the code later today).
\end{frame}

%% TODO: Add numerical Example from my own research


\section{This Class}
\subsection{Campelo's Lecture Notes}

\begin{frame}
  \frametitle{This week, on EDCS}
  
  \begin{itemize}
  \item \structure{DAE Chapter 05 -- Inferential Statistics:}\\ 
    Where we take our concepts of statistical intervals, and use them
    to extrapolate information about the population;
  \end{itemize}
\end{frame}

\begin{frame}
  \frametitle{Notes on the Lecture Notes 1}
  {\smaller
    \begin{itemize}
    \item Amos Nathan Tversky: Active Cognitive Psychologist on the
      field of human cognitive bias and handling of Risk. Humas are
      more irrational than we would feel comfortable with.
    \item \structure{Descriptive Statistics} vs \structure{inference
      statistics};
      \begin{itemize}
      \item Descriptive: Lets us characterize a population;
      \item Inference: Lets us make decisions about a population;
        Requires a \structure{degree of certainty} about
        characteristics of a population;
      \item When you are making a claim about research, in many cases
        you are not making a *description*, you are making a decision
        to claim something about the data. The decision is:\\
        Is your claim *safe*?\\
        Is your claim *supported by the data*?\\
        How likely is your claim to verify/refute the data?
      \end{itemize}
    \end{itemize}
  }
\end{frame}

\begin{frame}
  \frametitle{Notes on the Lecture Notes 1}
  {\smaller
    \begin{itemize}
    \item Hypothesis definition -- Examples
    \item Karl Popper -- Philosopher on the nature of truth. According
      to him, truth can only be obtained from
      observations. \alert{BUT} the way that we obtain truth from
      observations must be well regulated. (Falsification, Black Swan Fallacy)
    \item Discussion of how to formulate scientific hypothesis;
    \item Peas Example -- page 9: If mean of the sample is ``much
      bigger'' or ``much smaller'' than 500, then we refute the Null
      Hypothesis. But \alert{How Much} is much bigger or much smaller?
      This is the key question -- anyone has a good answer?
    \end{itemize}
  }
\end{frame}


%% TODO: Add comment from campelo Notes

%% TODO: Add some extra
\section{Extras}
\subsection{Extras}

\begin{frame}
  \frametitle{Suggested Reading}
  \begin{itemize}
  \item \structure{The Elements of Style(Willian Strunk Jr)} -- Recommended by Marco Antonio\\
    \url{http://www.amazon.com/Elements-Style-William-Strunk-Jr/dp/1557427283}
  \item \structure{How to do Good Research and get it Published (Eamonn Keogh)} -- Recommended by Mateus Cruz\\
    \url{http://www.cs.ucr.edu/~eamonn/Keogh_SIGKDD09_tutorial.pdf}
  \end{itemize}
\end{frame}



\end{document}
