\documentclass[t]{beamer}

% Load general definitions

\usepackage{amssymb,amsmath}
\usepackage{graphicx}
\usepackage{url}
\usepackage{color}
\usepackage{relsize}		% For \smaller
\usepackage{url}			% For \url
\usepackage{epstopdf}	% Included EPS files automatically converted to PDF to include with pdflatex

%For MindMaps
% \usepackage{tikz}%
% \usetikzlibrary{mindmap,trees,arrows}%

%%% Color Definitions %%%%%%%%%%%%%%%%%%%%%%%%%%%%%%%%%%%%%%%%%%%%%%%%%%%%%%%%%
%\definecolor{bordercol}{RGB}{40,40,40}
%\definecolor{headercol1}{RGB}{186,215,230}
%\definecolor{headercol2}{RGB}{80,80,80}
%\definecolor{headerfontcol}{RGB}{0,0,0}
%\definecolor{boxcolor}{RGB}{186,215,230}

%%% Save space in lists. Use this after the opening of the list %%%%%%%%%%%%%%%%
%\newcommand{\compresslist}{
%	\setlength{\itemsep}{1pt}
%	\setlength{\parskip}{0pt}
%	\setlength{\parsep}{0pt}
%}

%\setbeameroption{show notes on top}

% You should run 'pdflatex' TWICE, because of TOC issues.

% Rename this file.  A common temptation for first-time slide makers
% is to name it something like ``my_talk.tex'' or
% ``john_doe_talk.tex'' or even ``discrete_math_seminar_talk.tex''.
% You really won't like any of these titles the second time you give a
% talk.  Try naming your tex file something more descriptive, like
% ``riemann_hypothesis_short_proof_talk.tex''.  Even better (in case
% you recycle 99% of a talk, but still want to change a little, and
% retain copies of each), how about
% ``riemann_hypothesis_short_proof_MIT-Colloquium.2000-01-01.tex''?

\mode<presentation>
{
  % A tip: pick a theme you like first, and THEN modify the color theme, and then add math content.
  % Warsaw is the theme selected by default in Beamer's installation sample files.

  %%%%%%%%%%%%%%%%%%%%%%%%%%%% THEME
  %\usetheme{AnnArbor}
  %\usetheme{Antibes}
  %\usetheme{Bergen}
  %\usetheme{Berkeley}		% bem bacana - menu esquerdo
  %\usetheme{Berlin}
  %\usetheme{Boadilla}
  %\usetheme{boxes}
  %\usetheme{CambridgeUS}		% bem bacana - menu superior
  %\usetheme{Copenhagen}
  %\usetheme{Darmstadt}
  %\usetheme{default}
  %\usetheme{Dresden}
  \usetheme{Frankfurt}
  %\usetheme{Goettingen}
  %\usetheme{Hannover}		% bem bacana - menu esquerdo
  %\usetheme{Ilmenau}
  %\usetheme{JuanLesPins}
  %\usetheme{Luebeck}
  %\usetheme{Madrid}		%bacana
  %\usetheme{Malmoe}
  %\usetheme{Marburg}		% bem bacana - menu direito
  %\usetheme{Montpellier}
  %\usetheme{PaloAlto}		% bem bacana - menu esquerdo
  %\usetheme{Pittsburgh}
  %\usetheme{Rochester}		%bacana
  %\usetheme{Singapore}
  %\usetheme{Szeged}
  %\usetheme{Warsaw}

  %%%%%%%%%%%%%%%%%%%%%%%%%%%% COLOR THEME
  %\usecolortheme{albatross}		% azul escuro, massa
  %\usecolortheme{beetle}		% cinza, menu azul
  %\usecolortheme{crane}		% branco e amarelo, massa
  \usecolortheme{default}		% branco, azul clarinho
  %\usecolortheme{dolphin}		% azul e branco, legal
  %\usecolortheme{dove}			% cinza e branco, feio
  %\usecolortheme{fly}			% todo cinza, horrível
  %\usecolortheme{lily}			% parece o default
  %\usecolortheme{orchid}		% azul e branco, ok
  %\usecolortheme{rose}			% branco e violeta-claro, bonito
  %\usecolortheme{seagull}		% cinza, feio
  %\usecolortheme{seahorse}		% nhé, meio feio
  %\usecolortheme{sidebartab}		% Azul, branco, destaque na tab, interessante
  %\usecolortheme{structure}		% bichado
  %\usecolortheme{whale}		% Azul e branco, bem bonito

  %%%%%%%%%%%%%%%%%%%%%%%%%%%% OUTER THEME
  \useoutertheme{default}
  %\useoutertheme{infolines}
  %\useoutertheme{miniframes}
  %\useoutertheme{shadow}
  %\useoutertheme{sidebar}
  %\useoutertheme{smoothbars}
  %\useoutertheme{smoothtree}
  %\useoutertheme{split}
  %\useoutertheme{tree}

  %%%%%%%%%%%%%%%%%%%%%%%%%%%% INNER THEME
  \useinnertheme{circles}
  %\useinnertheme{default}
  %\useinnertheme{inmargin}
  %\useinnertheme{rectangles}
  %\useinnertheme{rounded}

  %%%%%%%%%%%%%%%%%%%%%%%%%%%%%%%%%%%

  \setbeamercovered{invisible} % or whatever (possibly just delete it)
  % To change behavior of \uncover from graying out to totally
  % invisible, can change \setbeamercovered to invisible instead of
  % transparent. apparently there are also 'dynamic' modes that make
  % the amount of graying depend on how long it'll take until the
  % thing is uncovered.

}


% Get rid of nav bar
\beamertemplatenavigationsymbolsempty

% Use short top
%\usepackage[headheight=12pt,footheight=12pt]{beamerthemeboxes}
%\addheadboxtemplate{\color{black}}{
%\hskip0.5cm
%\color{white}
%\insertshortauthor \ \ \ \ 
%\insertframenumber \ \ \ \ \ \ \ 
%\insertsection \ \ \ \ \ \ \ \ \ \ \ \ \ \ \ \ \  \insertsubsection
%\hskip0.5cm}
%\addheadboxtemplate{\color{black}}{
%\color{white}
%\ \ \ \ 
%\insertsection
%}
%\addheadboxtemplate{\color{black}}{
%\color{white}
%\ \ \ \ 
%\insertsubsection
%}

% Insert frame number at bottom of the page.
% \usefoottemplate{\hfil\tiny{\color{black!90}\insertframenumber}} 

\usepackage[english]{babel}
\usepackage[latin1]{inputenc}
\usepackage{subfigure}

\usepackage{times}
\usepackage[T1]{fontenc}


% Specific definitions
\title[]{Experiment Design for Computer Sciences}
\subtitle[]{Week 04 -- Inferential Statistics -- Course Notes}
\author[]{Claus Aranha\\{\footnotesize \url{http://conclave.cs.tsukuba.ac.jp/}}}
\institute{Computer Science Department}
\date{2016-05-13 (updated \today)}

\begin{document}
\section{Outline}
\subsection{Outline}
\begin{frame}
  \maketitle
\end{frame}

\begin{frame}
  \frametitle{Course Notes/Warnings}

  \begin{alertblock}{Team Project -- Don't procastinate on your team project!}
    \begin{itemize}
      \item I have so far only received 1 team e-mail;
      \item We are almost at the halfway point of the course! 
    \end{itemize}
  \end{alertblock}

  \vfill

  Send me an e-mail (caranha@cs.tsukuba.ac.jp) with the names, student
  IDs, and mail addresses of your team members.

  \vfill

  If you send me the topic of your project, I would be very happy too.
\end{frame}

\section{Last Class}
\subsection{Last Class}

\begin{frame}
  \frametitle{Review of Last Class}

  \begin{itemize}
  \item \structure{Concepts of Statistics}
    \begin{itemize}
    \item Goats, and why the world sometimes does not follow our intuition
    \item Difference between \structure{sample} and \structure{population}
    \item The \structure{Central Limit Theorem}
    \end{itemize} 

    \vfill

  \item \structure{DAE Chapter 03 -- Point Estimators}
    \begin{itemize}
    \item Estimating values for the \structure{population parameters} from the \structure{sample parameters};
    \item \structure{Bias} and \structure{variance} of estimators;
    \end{itemize}
    
    \vfill

  \item \structure{DAE Chapter 04 -- Interval Estimators}
    \begin{itemize}
    \item \structure{Confidence Interval} for a parameter
    \item How to interpret the Confidence Interval;
    \end{itemize}
  \end{itemize}
\end{frame}

\begin{frame}
  \frametitle{Review of Last Class -- Examples} 

  In Manaba, there was a ``Chapter04.R'' file with example code to
  calculate Confidence Intervals in R (and plot them).

  \bigskip

  Please take a look at the code and apply the calculations to your
  own data. (If we have time, we will review the code later today).
\end{frame}

%% TODO: Add numerical Example from my own research


\section{This Class}
\subsection{Campelo's Lecture Notes}

\begin{frame}
  \frametitle{This week, on EDCS}
  
  \begin{itemize}
  \item \structure{DAE Chapter 05 -- Inferential Statistics:}\\ 
    Where we take our concepts of statistical intervals, and use them
    to extrapolate information about the population;
  \end{itemize}
\end{frame}

\begin{frame}
  \frametitle{Notes on the Lecture Notes 1}
  {\smaller
    \begin{itemize}
    \item Amos Nathan Tversky: Active Cognitive Psychologist on the
      field of human cognitive bias and handling of Risk. Humas are
      more irrational than we would feel comfortable with.
    \item \structure{Descriptive Statistics} vs \structure{inference
      statistics};
      \begin{itemize}
      \item Descriptive: Lets us characterize a population;
      \item Inference: Lets us make decisions about a population;
        Requires a \structure{degree of certainty} about
        characteristics of a population;
      \item When you are making a claim about research, in many cases
        you are not making a *description*, you are making a decision
        to claim something about the data. The decision is:\\
        Is your claim *safe*?\\
        Is your claim *supported by the data*?\\
        How likely is your claim to verify/refute the data?
      \end{itemize}
    \end{itemize}
  }
\end{frame}

\begin{frame}
  \frametitle{Notes on the Lecture Notes 1}
  {\smaller
    \begin{itemize}
    \item Hypothesis definition -- Examples
    \item Karl Popper -- Philosopher on the nature of truth. According
      to him, truth can only be obtained from
      observations. \alert{BUT} the way that we obtain truth from
      observations must be well regulated. (Falsification, Black Swan Fallacy)
    \item Discussion of how to formulate scientific hypothesis;
    \item Peas Example -- page 9: If mean of the sample is ``much
      bigger'' or ``much smaller'' than 500, then we refute the Null
      Hypothesis. But \alert{How Much} is much bigger or much smaller?
      This is the key question -- anyone has a good answer?
    \item Anyway, the idea is that once we calculate this ``delta'',
      or this critical region, we can answer the question: Is the
      estimate value calculated from the sample inside this critical
      value? If it is, even if the value is the not the same as the
      hypothesis value, \structure{it is close enough} that we say we
      can't reject the null hypothesis.
    \end{itemize}
  }
\end{frame}

\begin{frame}
  \frametitle{Notes on the Lecture Notes 2}
  {\smaller
    \begin{itemize}
    \item Inferential errors, their definitions, and how we can
      control them;
    \item Type one error is a false positive, can be avoided with
      stricter tests;
    \item Type two error depends on the power of the test;
    \end{itemize}
  }
\end{frame}



%% TODO: Add comment from campelo Notes

%% TODO: Add some extra
\section{Extras}
\subsection{Extras}

\begin{frame}
  \frametitle{Suggested Reading}
  \begin{itemize}
  \item \structure{The Elements of Style(Willian Strunk Jr)} -- Recommended by Marco Antonio\\
    \url{http://www.amazon.com/Elements-Style-William-Strunk-Jr/dp/1557427283}
  \item \structure{How to do Good Research and get it Published (Eamonn Keogh)} -- Recommended by Mateus Cruz\\
    \url{http://www.cs.ucr.edu/~eamonn/Keogh_SIGKDD09_tutorial.pdf}
  \end{itemize}
\end{frame}



\end{document}
